\pdfminorversion=5

\documentclass[12pt]{article}
\usepackage[margin=1in]{geometry}
\usepackage{longtable}
\usepackage{bytefield}
\usepackage{color}
\usepackage[bookmarksopen=true]{hyperref}
\usepackage[pdftex]{graphicx}
%\usepackage{graphvizzz}  % <---- This needs to be included
\usepackage[utf8x]{inputenc} %utf8 %latin1
\usepackage{polski}
\usepackage{rotating}
\usepackage{listings}
\lstset{language=python} 
\usepackage{amsmath}
\usepackage{footnote}
\usepackage{xspace}
\usepackage{textcomp}

\title{Fizyka w Hello World Open 2014}
\author{Jacek Królikowski, Piotr Sokólski}
\date{\today}
\hypersetup{%
pdftitle={},
pdfauthor={Jacek Królikowski <jacek.krolikowski@almost-done.net>, Piotr Sokólski <p@pyetras.com>},
pdfsubject={},
pdfkeywords={}}

% Set up hyperlink colors
\definecolor{darkred}{rgb}{0.5,0,0}
\definecolor{darkgreen}{rgb}{0,0.3,0}
\definecolor{darkblue}{rgb}{0,0,0.5}
\definecolor{darkbrown}{rgb}{0.28,0.07,0.07}
\hypersetup{%
  colorlinks=true,
  citecolor=darkblue,
  urlcolor=darkgreen,
  linkcolor=darkred,
  menucolor=darkbrown}

\newcommand{\atm}{\emph{ATM}\xspace}

\makeatletter
\def\imod#1{\allowbreak\mkern10mu({\operator@font mod}\,\,#1)}
\makeatother

\begin{document}
\sloppy
\maketitle
\section{oznaczenia}
Jako, że będziemy używali w kodzie oznaczeń angielskich, a większość literek wywodzi się od angielskich słów, oznaczenia w obu językach:

\subsubsection{podstawowe}
$t$ - time - czas, wyrażony w intowych tickach \\
$s$ - distance - droga, wyrażona we flatach \\
$V$ - velocity - prędkość \\
$a$ - engine acceleration - przyspieszenie odsilnikowe \\
$e$ - engine power / maximal acceleration - ``moc'' silnika, maksymalne przyspieszenie jakie silnik może nadać samochodowi - w naszej pierwszej planszy ma wartość $0.2$  \\
$g$ - throtle - gaz \\
$b$ - drag deceleration - opór \\
$d$ - drag coefficient - współczynnik oporu - w naszej pierwszej planszy ma wartość $0.02$ \\
$c$ - net acceleration - przyspieszenie wypadkowe
\subsubsection{zaawansowane}
Te tutaj mogą się jeszcze zmienić, bo jeszcze nie wiemy, co rządzi kątem, zakrętami itd.\\
$\alpha$ - slip angle - kąt driftu\\
$r$ - bend radius - promień (faktyczny - z uwzględnieniem pasa itd) zakrętu\\
$m$ - mass - masa samochodu \\
$F_d$ - siła dośrodkowa \\
$F_t$ - maksymalna siła przyczepności opon \\
\section{wzory}
\subsection{jazda po prostej}
Wszystkie poniże wzory są chwilowe, należy je liczyć raczej na przestrzeni jednego ticku.

$V = \frac{ds}{dt}$ - prędkość chwilowa

$c = \frac{dV}{dt}$ - przyspieszenie wypadkowe

$c = a - b$ - wypadkowe przyspieszenie zależy od przyspieszenia odsilnikowego i od oporów

$a = e * g$ - przy czym $e$ nie jest znane, trzeba je wyliczyć

$b = V * d$ - opór jest wprost proporcjonalny do prędkości

Więc $c = e*g - V*d$. 

Dobrym podejściem do szybkiego tego mierzenia $d$ i $e$ jest wciśnięcie gazu do dechy na starcie i w pewnym momencie zdjęcie gazu na jeden tick. W czasie zdjęcia gazu na ten jeden tick możemy obliczyć $d$, a potem odwołując się do wcześniejszych wartości można policzyć $e$ już uwzględniając jakie były w jakimś tam momencie opory. Być może liczenie tego z pierwszego ticku przyspieszającego też będzie działało - trzeba sprawdzić (na przykład w tym excelu, którego mam pod windą).

Wzór na prędkość w chwili $x+1$: $$V_{x+1} = V_{x} + (e * g - V_x * d)$$

Wyprowadzenie wzoru na położenie gazu potrzebne do utrzymania prędkości $V_0$: 

$c=0$

$a=b$

$e * g = V_0 * d$

$g = \frac{V_0d}{e}$

\subsubsection{Wzory do określenia}
Przez to, że opór zmienia się w zależności od prędkości, nie mamy prostych wzorów na większość rzeczy - trzeba je będzie wyprowadzić (zakładając, że znamy $d$ i $e$). Pewnie wszystkie z tych wzorów będą zwarte, może będą zawierały jakieś całki.

\begin{itemize}
 \item zakładając stały poziom przepustnicy i mając daną prędkość początkową, funkcja drogi od czasu - do prognozowania położeń samochodów, przyspieszających czy zwalniających, i tak.
 \item zakładając stały poziom przepustnicy i mając daną prędkość początkową, funkcja prędkości od drogi. Do tego, żeby wiedzieć, kiedy zdejmować gaz przed zakrętem chociażby.
 \item pewnie jeszcze jakieś inne, ale na razie tylko te dwie powyżej wydają mi się potrzebne
\end{itemize}

\subsection{jazda po łuku}

Kąt, przy jakim wypadamy z trasy, jest stały, niezależny od prędkości i promienia skrętu, chociaż pewnie jest jedną ze stałych fizycznych. W naszym przypadku z zakrętu wypada się po przekroczeniu/osiągnięciu kąta 60\textdegree.

Najogólniej, wzór na ``stabilny'' kąt dla jednego promienia skrętu w zależności od prędkości jest $\alpha = A * V + B$, jeśli $\alpha>0$, $0$ wpp, gdzie $A$ i $B$ to pewne współczynniki. Ustalenie tych współczynników w zaleśności od kąta właśnie trwa, ale wydaje mi się, że $B$ jest stałe, tylko $A$ się zmienia.

Ciekawa sprawa - zmiana slip angle też ma pewną ``inercję'' - nawet jeśli kąt stabilny to na przykład 45\textdegree, to jeśli kąt w szybkim zakręcie przybiera od zera, to 45\textdegree nie będzie ``nieprzekraczalną'' asymptotą, tylko kąt przekroczy to 45\textdegree, potem się wachnie i wróci do tych 45\textdegree. Trudno mi to na szyko zdefiniować, ale dobrze to widać na wykresie, który dodam później.

\subsubsection{inne przemyślenia nie mające związku z rzeczywistością}

Jest wzór na siłę dośrodkową - siłę skierowaną środka łuku potrzebną, żeby coś zasuwało z jakąś prędkością po zadanym łuku. 

$F_d = \frac{mV^2}{r}$

Zakładając, że przyczepność kół jest stała, można się domyślać, że samochód będzie miał pewną maksymalną siłę przyczepności (?) jaką mogą zapewnić opony, powyżej której, będzie driftował - $F_t$. Kiedy wyliczone $F_d > F_t$ - będzie zaczynał driftować. Niekoniecznie, bo być może część przyczepności będzie (jak w rzeczywistości) ``pochłaniało'' napędzanie samochodu $F_n = a * m$ i trzeba policzyć wypadkową z prostopadłych $F_d$ i $F_n$, która będzie przekraczała $F_t$ kiedy przyczepność będzie zrywana. To wszystko domysły, zobaczymy.

Podobnie jestem przekonany, że położenie \textit{guide flag} też ma kluczowe znaczenie - im bardziej centralne położenie pod samochodem, tym jest stabilniejszy w zakrętach - zobaczymy.

Generalnie wydaje mi się, że w tym momencie trzeba zrobić eksperymenty, przy jakich prędkościach samochód zaczyna driftować w zależności od promienia zakrętu, ale najlepiej przy minimalnym przyspieszeniu, żeby wyeliminować wpływ przyspieszenia na wzór... :S

\end{document}
