\pdfminorversion=5

\documentclass[12pt]{article}
\usepackage[margin=1in]{geometry}
\usepackage{longtable}
\usepackage{bytefield}
\usepackage{color}
\usepackage[bookmarksopen=true]{hyperref}
\usepackage[pdftex]{graphicx}
%\usepackage{graphvizzz}  % <---- This needs to be included
\usepackage[utf8x]{inputenc} %utf8 %latin1
\usepackage{polski}
\usepackage{rotating}
\usepackage{listings}
\lstset{language=python} 
\usepackage{amsmath}
\usepackage{footnote}
\usepackage{xspace}
\usepackage{textcomp}

\title{Fizyka w Hello World Open 2014}
\author{Jacek Królikowski, Piotr Sokólski}
\date{\today}
\hypersetup{%
pdftitle={},
pdfauthor={Jacek Królikowski <jacek.krolikowski@almost-done.net>, Piotr Sokólski <p@pyetras.com>},
pdfsubject={},
pdfkeywords={}}

% Set up hyperlink colors
\definecolor{darkred}{rgb}{0.5,0,0}
\definecolor{darkgreen}{rgb}{0,0.3,0}
\definecolor{darkblue}{rgb}{0,0,0.5}
\definecolor{darkbrown}{rgb}{0.28,0.07,0.07}
\hypersetup{%
  colorlinks=true,
  citecolor=darkblue,
  urlcolor=darkgreen,
  linkcolor=darkred,
  menucolor=darkbrown}

\newcommand{\atm}{\emph{ATM}\xspace}

\makeatletter
\def\imod#1{\allowbreak\mkern10mu({\operator@font mod}\,\,#1)}
\makeatother

\makeatletter
\def\imax#1{\allowbreak({\operator@font max}\,\,#1)}
\makeatother


\begin{document}
\sloppy
\maketitle
\section{oznaczenia}
Jako, że będziemy używali w kodzie oznaczeń angielskich, a większość literek wywodzi się od angielskich słów, oznaczenia w obu językach:

\subsubsection{podstawowe}
$t$ - time - czas, wyrażony w intowych tickach \\
$s$ - distance - droga, wyrażona we floatach \\
$V$ - velocity - prędkość \\
$a$ - engine acceleration - przyspieszenie odsilnikowe \\
$e$ - engine power / maximal acceleration - ``moc'' silnika, maksymalne przyspieszenie jakie silnik może nadać samochodowi - w naszej pierwszej planszy ma wartość $0.2$  \\
$g$ - throtle - gaz \\
$b$ - drag deceleration - opór \\
$d$ - drag coefficient - współczynnik oporu - w naszej pierwszej planszy ma wartość $0.02$ \\
$c$ - net acceleration - przyspieszenie wypadkowe
\subsubsection{zaawansowane}
Te tutaj mogą się jeszcze zmienić, bo jeszcze nie wiemy, co rządzi kątem, zakrętami itd.\\
$\alpha$ - slip angle - kąt driftu - jeśli odchylamy się w prawo od kierunku toru jest dodatni, jeśli w lewo - ujemny\\
$r$ - bend radius - promień (faktyczny - z uwzględnieniem pasa itd) zakrętu\\
$m$ - mass - masa samochodu \\
$\omega$ - prędkość kątowa \\
$M$ - wypadkowy moment siły \\
$M_p$ - prostujący moment siły \\
$p$ - współczynnik prostujący, u nas równy $0.00125$\\
$M_d$ - dampening - tłumiący moment siły \\
$\zeta$ - współczynnik tłumienia\\
$M_c$ - odśrodkowy moment siły \\
$F_t$ - maksymalna siła przyczepności opon \\
\section{wzory}
\subsection{jazda po prostej}
Wszystkie poniże wzory są chwilowe, należy je liczyć raczej na przestrzeni jednego ticku.

$V = \frac{ds}{dt}$ - prędkość chwilowa

$c = \frac{dV}{dt}$ - przyspieszenie wypadkowe

$c = a - b$ - wypadkowe przyspieszenie zależy od przyspieszenia odsilnikowego i od oporów

$a = e \cdot g$ - przy czym $e$ nie jest znane, trzeba je wyliczyć

$b = V \cdot d$ - opór jest wprost proporcjonalny do prędkości

Więc $c = e\cdot g - V\cdot d$. 

Dobrym podejściem do szybkiego tego mierzenia $d$ i $e$ jest wciśnięcie gazu do dechy na starcie i w pewnym momencie zdjęcie gazu na jeden tick. W czasie zdjęcia gazu na ten jeden tick możemy obliczyć $d$, a potem odwołując się do wcześniejszych wartości można policzyć $e$ już uwzględniając jakie były w jakimś tam momencie opory. Być może liczenie tego z pierwszego ticku przyspieszającego też będzie działało - trzeba sprawdzić (na przykład w tym excelu, którego mam pod windą).

Wzór na prędkość w chwili $x+1$: $$V_{x+1} = V_{x} + (e \cdot  g - V_x \cdot  d)$$

Wyprowadzenie wzoru na położenie gazu potrzebne do utrzymania prędkości $V_0$: 

$c=0$

$a=b$

$e \cdot  g = V_0 \cdot  d$

$g = \frac{V_0d}{e}$

\subsubsection{Wzory do określenia}
Przez to, że opór zmienia się w zależności od prędkości, nie mamy prostych wzorów na większość rzeczy - trzeba je będzie wyprowadzić (zakładając, że znamy $d$ i $e$). Pewnie wszystkie z tych wzorów będą zwarte, może będą zawierały jakieś całki.

\begin{itemize}
 \item zakładając stały poziom przepustnicy i mając daną prędkość początkową, funkcja drogi od czasu - do prognozowania położeń samochodów, przyspieszających czy zwalniających, i tak.
 \item zakładając stały poziom przepustnicy i mając daną prędkość początkową, funkcja prędkości od drogi. Do tego, żeby wiedzieć, kiedy zdejmować gaz przed zakrętem chociażby.
 \item pewnie jeszcze jakieś inne, ale na razie tylko te dwie powyżej wydają mi się potrzebne
\end{itemize}

\subsection{jazda po łuku}

Kąt, przy jakim wypadamy z trasy, jest stały, niezależny od prędkości i promienia skrętu, chociaż pewnie jest jedną ze stałych fizycznych. W naszym przypadku z zakrętu wypada się po przekroczeniu/osiągnięciu kąta 60\textdegree.

$F_d = \frac{mV^2}{r}$

\subsubsection{slip angle dynamics}
Slip angle daje się opisać uproszczonymi wzorami na fizyczny ruch obrotowy. Żeby zmniejszyć liczbę współczynników wzory które będę podawał niekoniecznie są prawdziwe IRL, ale działają w HWO i mają minimalną potrzebną liczbę współczynników - przyspieszenia będą się mieszać z siłami :).

Działa zasada zachowania momentu pędu - prędkość obrotowa (zmiana kąta w czasie) nie zmieniałaby się, gdyby nie działanie momentów siły (przyspieszeń kątowych). W naszym przypadku działają 3 takie siły/przyspieszenia:
\begin{itemize}
  \item $M_p$ - przyspieszenie prostujące (biorące się z np oporów) 
 \item $M_d$ -  dampening  - spowalniające ruch obrotowy, niezależnie od jego kierunku
 \item $M_c$ -przyspieszenie odśrodkowe - działa na zakrętach, skierowane zawsze do zewnątrz zakrętów
\end{itemize}

Dwie pierwsze z powyższych sił mają dużo wspólnego z ``lekko przytłumionym oscylatorem harmonicznym'', o którym jest dużo napisane na wikipedii.

Poniższe wzory działają (znowu) z ticku na tick - wartości są aktualizowane na podstawie wartości z poprzedniego ticku

$\alpha_{t+1} = \alpha_{t} + \omega$

$\omega_{t+1} = \omega_{t} + M$

$M = M_p + M_d + M_c$

$M_p = - v \cdot  \alpha \cdot  p$

$M_d = - \omega \cdot  \zeta$

$M_c$ - dokładny wzór jest nieznany, ale jest zależne tylko od promienia $r$ i prędkości $v$ - nie potrafimy go liczyć dokładnie.

Znaczy to, że możemy z praktycznie dowolną precyzją liczyć stan samochodu tylko kiedy jedzie po prostej - w zakrętach musimy się na razie zdać na przybliżenia.

Wykres $M_c$ ma kształt podobny do $M_c \approx max(\frac{V^2}{r} - F_t, 0) \cdot  m$ (z dokładnością do znaku, powinien być ujemny na zakrętach o ujemnym kącie), ale nie udało się uzyskać współczyników, które dałyby bliski zeru błąd. Można jednak założyć, że $M_c = f(V, r)$ i na tej podstawie tablicować i przbyliżać

Podobnie jak ze wzorami na prostej przydałoby się wyprowadzić wzory na kąt i prędkość kątową od czasu, dla (stałej prędkości albo stałego przyspieszenia) i stałego $r$. Dzięki nim możnaby szybko sprawdzać, z jaką prędkością można wchodzić w nadchodzący zakręt.

\subsubsection{inne przemyślenia nie mające związku z rzeczywistością}
Jestem przekonany, że położenie \textit{guide flag} też ma kluczowe znaczenie - im bardziej centralne położenie pod samochodem, tym jest stabilniejszy w zakrętach - zobaczymy.

\end{document}
